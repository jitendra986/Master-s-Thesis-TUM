\usepackage[utf8]{inputenc}
\usepackage[T1]{fontenc} % Zeichensatzkodierung

\usepackage{calc} % Berechnungen
\usepackage[english]{babel} % Deutsche Lokalisierung
\usepackage{graphicx} % Grafiken
\usepackage[export]{adjustbox}
\usepackage[absolute, overlay]{textpos} % Positionierung
\usepackage{bbm}

%\usepackage{paralist}
\usepackage[linesnumbered]{algorithm2e}
\usepackage{amsmath}
\usepackage{colortbl}
%\usepackage{bbold}
\usepackage[hang,small]{subfigure} 
\usepackage{graphicx}
\usepackage[style=numeric-comp,autocite=superscript,backend=bibtex]{biblatex}
\setbeamertemplate{bibliography item}{}
\usepackage{lipsum}

% Silbentrennung:
\usepackage{hyphenat}
%\tolerance 2414
%\hbadness 2414
%\emergencystretch 1.5em
%\hfuzz 0.3pt
%\widowpenalty=10000     % Hurenkinder
%\clubpenalty=10000      % Schusterjungen
%\vfuzz \hfuzz

% Euro-Symbol:
\usepackage[gen]{eurosym}
\DeclareUnicodeCharacter{20AC}{\euro{}}

% Schriftart Helvetica:
%\useinnertheme{default}
%\usefonttheme{professionalfonts}
%\usefonttheme{serif}
%\renewcommand\mathfamilydefault{\rmdefault}
\usefonttheme[onlymath]{serif}
\renewcommand\sfdefault{cmbr}

\usepackage[scaled]{helvet}
\renewcommand{\familydefault}{\sfdefault}
%\usepackage{mathptmx} % skalierbare Formelschriften
\usepackage{tabularx}
\usepackage{multicol} % mehrspaltiger Text
\usepackage{tikz}
\usetikzlibrary{arrows, shapes, shapes.multipart, trees, positioning,
    backgrounds, fit, matrix}

% Diagramme:
\usepackage{pgfplots}
\pgfplotsset{compat=default}

% Erweiterbare Fusszeile:
\newcommand{\PraesentationSchriftgroesseSehrGross}{\fontsize{30}{45}}
\newcommand{\PraesentationSchriftgroesseGross}{\fontsize{22}{33}}
\newcommand{\PraesentationSchriftgroesseNormal}{\fontsize{16}{29}}
\newcommand{\PraesentationSchriftgroesseKlein}{\fontsize{12}{18}}
\newcommand{\PraesentationSchriftgroesseDreizeiler}{\fontsize{7}{10}}
\newcommand{\PraesentationSchriftgroesseAufzaehlungszeichen}{\fontsize{10}{15}}
\newcommand{\PraesentationAbstandAbsatz}{22.1pt}
\newcommand{\PraesentationPositionKorrekturOben}{-0.75cm}
\newcommand{\PraesentationBeispieleSchriftgroessen}{30 | 22 | 16 | 12}

    % Blautöne:
\definecolor{TUMBlau}{RGB}{0,101,189} % Pantone 300
\definecolor{TUMBlauDunkel}{RGB}{0,82,147} % Pantone 301
\definecolor{TUMBlauHell}{RGB}{152,198,234} % Pantone 283
\definecolor{TUMBlauMittel}{RGB}{100,160,200} % Pantone 542

    % Hervorhebung:
\definecolor{TUMElfenbein}{RGB}{218,215,203} % Pantone 7527 -Elfenbein
\definecolor{TUMGruen}{RGB}{162,173,0} % Pantone 383 - Grün
\definecolor{TUMOrange}{RGB}{227,114,34} % Pantone 158 - Orange
\definecolor{TUMGrau}{gray}{0.6} % Grau 60%


%%%%%%%%%%%%%%%%%%%%%%%%%%%%%%%%%%%%%%%%%%%%%%%%%%%%%%%
% Personenspezifische Informationen
%%%%%%%%%%%%%%%%%%%%%%%%%%%%%%%%%%%%%%%%%%%%%%%%%%%%%%%%
% Für die Person anpassen:
\newcommand{\PersonAdresse}{%
    @Adresse@\\%
    @Plz@~\PersonStadt%
}
\newcommand{\PersonTelefon}{@Telefon@}
\newcommand{\PersonFax}{@Fax@}
\newcommand{\PersonEmail}{@E-Mail@}
\newcommand{\PersonWebseite}{@Web@}

\newcommand{\FakultaetAnsprechpartner}{@Ansprechpartner@}
% Fakultät:
\newcommand{\EinstellungBankName}{Bayerische Landesbank}
\newcommand{\EinstellungBankIBAN}{DE10700500000000024866}
\newcommand{\EinstellungBankBIC}{BYLADEMM}
\newcommand{\EinstellungSteuernummer}{143/241/80037}
\newcommand{\EinstellungUmsatzsteuerIdentifikationsnummer}{DE811193231}

\hyphenation{} % eigene Silbentrennung

\newcommand{\suchthat}{\;\ifnum\currentgrouptype=16 \middle\fi|\;}
\newcommand{\norm}[1]{\left\Vert#1\right\Vert}
\newcommand{\abs}[1]{\left\vert#1\right\vert}
\newcommand{\iprod}[1]{\left\langle#1\right\rangle}
\newcommand{\eps}{\varepsilon}
\newcommand{\BibLaTeX}{{Bib}\LaTeX~}
\newcommand{\memoir}{\text{Memoir}~}
\newcommand{\dd}{\mathrm{d}}
\newcommand{\Ex}{\mathop{{}\mathbb{E}}}%\mathop{{}\mathbb{E}}
\newcommand{\I}{\pmb{\mathbb{I}}}%\mathop{{}\mathbb{E}}
\newcommand{\II}[2]{{\mathbbm{I}_{#1}\left[#2\right]}}
\newcommand{\cov}{\mathrm{Cov}}
\newcommand{\var}{\mathrm{Var}}
\newcommand{\deter}{\mathrm{det}}
\newcommand{\tran}{\mathsf{T}}
\newcommand{\matlab}{$\text{MATLAB}^{\text{\textregistered}}$~}
\newcommand{\defin}{\textit}
\newcommand{\intd}{\text{d}}
\newcommand{\N}{\mathrm{N}}
\newcommand{\logN}{\mathrm{logN}}
\newcommand{\diver}{\mathrm{div}}
\newcommand{\ve}[1]{{\boldsymbol{#1}}}
\DeclareMathOperator{\argmin}{argmin}
\DeclareMathOperator{\ble}{BLE}
\DeclareMathOperator{\enkfip}{EnKF}
\DeclareMathOperator{\E}{\mathbb{E}}
\DeclareMathOperator{\Cov}{Cov}
\DeclareMathOperator{\Var}{Var}
\DeclareMathOperator{\iid}{i.i.d.}
\newcommand{\highlight}[1]{\colorbox{TUMBlauHell}{$\displaystyle #1$}}
